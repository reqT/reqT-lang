%!TEX encoding = UTF-8 Unicode
\documentclass[a4paper,oneside]{article}

\usepackage[top=18mm,bottom=3mm, hmargin=10mm,landscape]{geometry}

\usepackage[utf8]{inputenc}
\usepackage[T1]{fontenc}

\usepackage{tgtermes}
\usepackage{lmodern}
\usepackage[scaled=0.9]{beramono} % inconsolata or beramono ???
\usepackage{microtype} % Slightly tweak font spacing for aesthetics

\usepackage{fancyhdr}
\pagestyle{fancy}
\chead{\url{https://github.com/reqT/reqT-lang/blob/main/docs/reqT-quickref-GENERATED.tex}}
\lhead{QuickRef reqT v4}
\rhead{Compiled \today}

\usepackage{hyperref}
\hypersetup{colorlinks=true, linkcolor=blue, urlcolor=blue}
\usepackage[usenames,dvipsnames,svgnames,table]{xcolor}
\definecolor{entityColor}{RGB}{0,100,200}
\definecolor{attributeColor}{RGB}{0,100,50}
\definecolor{relationColor}{RGB}{160,0,30}

\usepackage{listings}
\lstdefinestyle{reqT}{
  %belowcaptionskip=1\baselineskip,
  breaklines=true,
  %showstringspaces=false,
  showspaces=false,
  %breakatwhitespace=true,
  basicstyle=\ttfamily\small,
  emph={Actor,App,Barrier,Breakpoint,Class,Component,Configuration,Data,Design,Domain,Epic,Event,Feature,Field,Function,Goal,Idea,Image,Interface,Issue,Item,Label,Member,Module,Product,Prototype,Quality,Relationship,Release,Req,Resource,Risk,Scenario,Screen,Section,Service,Stakeholder,State,Story,System,Target,Task,Term,Test,UseCase,User,Variant,VariationPoint,WorkPackage},
  emphstyle=\bfseries\color{entityColor},
  emph={[2]binds,deprecates,excludes,has,helps,hurts,impacts,implements,interactsWith,is,precedes,relatesTo,requires,verifies,Binds,Deprecates,Excludes,Has,Helps,Hurts,Impacts,Implements,InteractsWith,Is,Precedes,RelatesTo,Requires,Verifies},
  emphstyle={[2]\bfseries\color{relationColor}},
  emph={[3]Benefit,Capacity,Comment,Constraints,Cost,Damage,Deprecated,Example,Expectation,Failure,Frequency,Gist,Input,Location,Max,Min,Order,Output,Prio,Probability,Profit,Spec,Text,Title,Value,Why},
  emphstyle={[3]\bfseries\color{attributeColor}},  
}

\lstset{style=reqT}
\usepackage{multicol}

\setlength\parindent{0em}
\setlength\headsep{1em}
\setlength\footskip{0em}
\usepackage{titlesec}
  \titlespacing{\section}{0pt}{3.5pt}{2pt}
  \titlespacing{\subsection}{0pt}{3.5pt}{2pt}
  \titlespacing{\subsubsection}{0pt}{3pt}{2pt}

\usepackage{titlesec}

\titleformat*{\section}{\normalfont\fontsize{12}{15}\bfseries}

\titleformat*{\subsection}{\normalfont\fontsize{10}{12}\bfseries}

\usepackage{graphicx}

\pagenumbering{gobble}

\renewcommand{\rmdefault}{\sfdefault}

\newcommand\Concept[2]{\hangindent=1em\lstinline+#1+ #2}

\begin{document}

\fontsize{9.1}{11}\selectfont

\begin{multicols*}{4}
\raggedright

\section*{What is reqT?}
The reqT requirements modelling language 
helps you structure requirements into semi-formal 
natural-language models using 
common requirements engineering concepts.


\section*{reqT Markdown syntax}
A reqT model in markdown syntax starts with \lstinline+*+ followed by an element and a colon and an optional relation followed by an indented list of sub-elements.

\begin{lstlisting}
* Feature: helloWorld has
  * Spec: Show informal greeting.
  * Prio: 1

\end{lstlisting}

\section*{reqT Metamodel}

A model is a sequence of elements. 
An element can be a node or a relation. 
A node can be an entity or an attribute. 
An entity has a type and an id. 
An attribute has a type and a value. 
An attribute can be a string attribute or an integer attribute. 
A relation connects an entity to a sub-model via a relation type.


\noindent\hspace*{-2.1em}\includegraphics[width=8.2cm]{metamodel-Elem-GENERATED.pdf}

\section*{\texttt{EntType}.values}
\subsection*{\underline{\texttt{\textit{{\textcolor{gray}{EntGroup.}\textcolor{black}{GeneralEnt}}\textcolor{gray}{.types}}}}}
\Concept{Epic}{A coherent collection of features, stories, use cases or issues. A large part of a release.}

\Concept{Feature}{A releasable characteristic of a product. A high-level, coherent bundle of requirements.}

\Concept{Goal}{An intention of a stakeholder or desired system property.}

\Concept{Idea}{A concept or thought, potentially interesting.}

\Concept{Image}{A visual representation, picture or diagram.}

\Concept{Interface}{A way to interact with a system.}

\Concept{Issue}{Something to be fixed or work to do.}

\Concept{Item}{An article in a collection, enumeration, or series.}

\Concept{Label}{A descriptive tag used to classify something.}

\Concept{Req}{Something needed or wanted. An abstract term denoting any type of information relevant to the (specification of) intentions behind system development. Short for requirement.}

\Concept{Section}{A part of a requirements document.}

\Concept{Term}{A word or group of words having a particular meaning in a particular domain.}

\Concept{Test}{A procedure to check if requirements are met.}

\Concept{WorkPackage}{A coherent collection of (development) activities.}


\subsection*{\underline{\texttt{\textit{{\textcolor{gray}{EntGroup.}\textcolor{black}{ContextEnt}}\textcolor{gray}{.types}}}}}
\Concept{Actor}{A role played by a user or external system that interacts with the system (product, app, or service) under development.}

\Concept{App}{A computer program, or group of programs designed for end users, normally with a graphical user interface. Short for application.}

\Concept{Domain}{The application area of a product with its surrounding entities, e.g. users or other systems.}

\Concept{Product}{An artifact offered to users or customers, e.g. an app, service or  embedded system.}

\Concept{Release}{A specific version of a product offered to end users at a specific time.}

\Concept{Resource}{A capability of, or support for product development, e.g. a development team or some testing equipment.}

\Concept{Risk}{Something negative that may happen.}

\Concept{Scenario}{A narrative of foreseeable interactions of user roles (actors) and the system (product, app, or service) under development..}

\Concept{Service}{System use that provides value to stakeholders. System actions that stakeholders are willing to pay for.}

\Concept{Stakeholder}{A role, person or legal entity with a stake in the development or operation of a product.}

\Concept{System}{A set of software or hardware components interacting with users or systems.}

\Concept{User}{A human interacting with a system.}


\subsection*{\underline{\texttt{\textit{{\textcolor{gray}{EntGroup.}\textcolor{black}{DataEnt}}\textcolor{gray}{.types}}}}}
\Concept{Class}{An extensible template for creating objects. A set of objects with certain attributes in common. A category.}

\Concept{Data}{A data entity, type, class, or record stored or processed by a system.}

\Concept{Field}{A data attribute that is part of another entity, such as a class.}

\Concept{Member}{A data entity that is part of another entity. More specific alternatives: field, function.}

\Concept{Relationship}{A specific way that data types are connected.}


\subsection*{\underline{\texttt{\textit{{\textcolor{gray}{EntGroup.}\textcolor{black}{FunctionalEnt}}\textcolor{gray}{.types}}}}}
\Concept{Event}{Something that can happen in the domain or in the system.}

\Concept{Function}{A description of how input is mapped to output. A capability of a system to do something specific.}

\Concept{State}{A mode or condition of something in the domain or in the system. A configuration of data.}

\Concept{Story}{A description of what a user wants in order to achieve a goal. Short for user story.}

\Concept{Task}{A piece of work by users, potentially supported by a system. Short for user task}

\Concept{UseCase}{A goal-fulfilling interaction between users and a product in a specific usage context.}


\subsection*{\underline{\texttt{\textit{{\textcolor{gray}{EntGroup.}\textcolor{black}{QualityEnt}}\textcolor{gray}{.types}}}}}
\Concept{Barrier}{Something that makes it difficult to achieve a goal or a higher quality level.}

\Concept{Breakpoint}{A point of change, representing an important shift in the relation between quality and benefit.}

\Concept{Quality}{An aspect of system quality, distinguishing characteristic or degree of goodness.}

\Concept{Target}{A desired quality level or quality goal.}


\subsection*{\underline{\texttt{\textit{{\textcolor{gray}{EntGroup.}\textcolor{black}{DesignEnt}}\textcolor{gray}{.types}}}}}
\Concept{Component}{A composable part of a system architecture. A reusable, interchangeable system unit or functionality.}

\Concept{Design}{A specific realization. A description of an implementation.}

\Concept{Module}{A collection of coherent functions and interfaces.}

\Concept{Prototype}{A mockup or system with limited functionality to demonstrate a design idea.}

\Concept{Screen}{A design of (a part of) a user interface.}


\subsection*{\underline{\texttt{\textit{{\textcolor{gray}{EntGroup.}\textcolor{black}{VariabilityEnt}}\textcolor{gray}{.types}}}}}
\Concept{Configuration}{A specific combination of variants.}

\Concept{Variant}{An object or system property that can be chosen from a set of options.}

\Concept{VariationPoint}{An opportunity of choice among variants.}



\section*{reqT Scala DSL constructors}
\lstinline+EntType+, \lstinline+StrAttrType+ and \lstinline+IntAttrType+ enums have apply-methods that construct \lstinline+Ent+, \lstinline+StrAttr+ and \lstinline+IntAttr+ instances respectively. Each instance of \lstinline+Ent+ has lower-case relation constructors (see \lstinline+enum relType+ on next page):
\begin{lstlisting}
Model(
  Feature("helloWorld").has(
    Spec("Show informal greeting."),
    Prio(1)))

\end{lstlisting}


\section*{reqT Scala case classes}
Each constructor instantiate the metamodel classes using nested Scala case class structures:
\begin{lstlisting}
Model(
  Rel(Ent(Feature,"helloWorld"),
    Has, Model(
      StrAttr(Spec,
        "Show informal greeting."),
      IntAttr(Prio,1))))

\end{lstlisting}

\end{multicols*}
%\raggedcolumns
\begin{multicols*}{4}
\raggedright

%\vfill\null\columnbreak

\section*{\texttt{StrAttrType.values}}
\Concept{Comment}{A note with a remark or a discussion on an entity.}

\Concept{Constraints}{A collection of propositions that constrain a solution space or restrict possible attribute values.}

\Concept{Deprecated}{A description of why an entity should be avoided, often because it is superseded by another entity, as indicated by a 'deprecates' relation.}

\Concept{Example}{A description that illustrates some entity by a typical instance.}

\Concept{Expectation}{A required output of a test in order to be counted as passed.}

\Concept{Failure}{A description of a runtime error that prevents the normal execution of a system.}

\Concept{Gist}{A short and simple description. A summary capturing the essence of an entity.}

\Concept{Input}{Data consumed by an entity, }

\Concept{Location}{A location of a resource such as a web address or a path to a file of persistent data.}

\Concept{Output}{Data produced by an entity, e.g. a function or a test.}

\Concept{Spec}{A definition of an entity. Short for specification}

\Concept{Text}{An paragraph or general description.}

\Concept{Title}{A general or descriptive heading.}

\Concept{Why}{A description of intention or rationale of an entity.}

%\vfill\null\columnbreak
\section*{\texttt{IntAttrType.values}}
\Concept{Benefit}{A characterization of a good or helpful result or effect (e.g. of a feature).}

\Concept{Capacity}{An amount that can be held or contained (e.g. by a resource).}

\Concept{Cost}{An expenditure of something, such as time or effort, necessary if implementing an entity.}

\Concept{Damage}{A characterization of the negative consequences if some entity (e.g. a risk) occurs.}

\Concept{Frequency}{A number of occurrences per time unit. }

\Concept{Max}{A maximum estimated or assigned value.}

\Concept{Min}{A minimum estimated or assigned value.}

\Concept{Order}{An ordinal number (1st, 2nd, ...).}

\Concept{Prio}{A level of importance of an entity. Short for priority.}

\Concept{Probability}{A likelihood expressed as whole percentages that something (e.g. a risk) occurs.}

\Concept{Profit}{A gain or return of some entity, e.g. in monetary terms.}

\Concept{Value}{Some general integer value.}
 

\section*{\texttt{RelType.values}}

\subsection*{\underline{\texttt{\textit{{\textcolor{gray}{RelGroup.}\textcolor{black}{GeneralRel}}\textcolor{gray}{.types}}}}}
\Concept{Deprecates}{Makes outdated. An entity deprecates (supersedes) another entity.}

\Concept{Has}{Expresses containment, substructure, composition or aggregation. One entity contains another.}

\Concept{Impacts}{Some unspecific influence. A new feature impacts an existing component.}

\Concept{RelatesTo}{Some general, unspecific relation to another entity.}


\subsection*{\underline{\texttt{\textit{{\textcolor{gray}{RelGroup.}\textcolor{black}{ClassRel}}\textcolor{gray}{.types}}}}}
\Concept{Is}{One entity inherits properties of another entity. A specialization, extension or subtype relation. }


\subsection*{\underline{\texttt{\textit{{\textcolor{gray}{RelGroup.}\textcolor{black}{ContextRel}}\textcolor{gray}{.types}}}}}
\Concept{InteractsWith}{A communication relation. A user interacts with an interface.}


\subsection*{\underline{\texttt{\textit{{\textcolor{gray}{RelGroup.}\textcolor{black}{DependencyRel}}\textcolor{gray}{.types}}}}}
\Concept{Excludes}{Prevents an entity combination. One feature excludes another in a release.}

\Concept{Implements}{Realisation of an entity, e.g. a component implements a feature.}

\Concept{Precedes}{Temporal ordering. A feature precedes (should be implemented before) another feature.}

\Concept{Requires}{A requested combination. One function requires that a another function is implemented.}

\Concept{Verifies}{Gives evidence of correctness. A test verifies the implementation of a feature.}


\subsection*{\underline{\texttt{\textit{{\textcolor{gray}{RelGroup.}\textcolor{black}{GoalRel}}\textcolor{gray}{.types}}}}}
\Concept{Helps}{Positive influence. A goal supports the fulfillment of another goal.}

\Concept{Hurts}{Negative influence. A goal hinders another goal.}


\subsection*{\underline{\texttt{\textit{{\textcolor{gray}{RelGroup.}\textcolor{black}{VariabilityRel}}\textcolor{gray}{.types}}}}}
\Concept{Binds}{Ties a value to an option. A configuration binds a variation point.}


\end{multicols*}

\end{document}